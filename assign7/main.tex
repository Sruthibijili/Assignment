\let\negmedspace\undefined
\let\negthickspace\undefined
\documentclass[journal]{IEEEtran}
\usepackage[a5paper, margin=10mm, onecolumn]{geometry}
%\usepackage{lmodern} % Ensure lmodern is loaded for pdflatex
\usepackage{tfrupee} % Include tfrupee package

\setlength{\headheight}{1cm} % Set the height of the header box
\setlength{\headsep}{0mm}     % Set the distance between the header box and the top of the text

\usepackage{gvv-book}
\usepackage{gvv}
\usepackage{cite}
\usepackage{amsmath,amssymb,amsfonts,amsthm}
\usepackage{algorithmic}
\usepackage{graphicx}
\usepackage{textcomp}
\usepackage{xcolor}
\usepackage{txfonts}
\usepackage{listings}
\usepackage{enumitem}
\usepackage{mathtools}
\usepackage{gensymb}
\usepackage{comment}
\usepackage[breaklinks=true]{hyperref}
\usepackage{tkz-euclide} 
\usepackage{listings}
% \usepackage{gvv}                                        
\def\inputGnumericTable{}                                 
\usepackage[latin1]{inputenc}                                
\usepackage{color}                                            
\usepackage{array}                                            
\usepackage{longtable}                                       
\usepackage{calc}                                             
\usepackage{multirow}                                         
\usepackage{hhline}                                           
\usepackage{ifthen}                                           
\usepackage{lscape}
\begin{document}

\bibliographystyle{IEEEtran}
\vspace{3cm}

\title{02-09-2020 shift-1-16-25}
\author{EE24BTECH11060 - Sruthi Bijili}
% \maketitle
% \newpage
% \bigskip
{\let\newpage\relax\maketitle}

\renewcommand{\thefigure}{\theenumi}
\renewcommand{\thetable}{\theenumi}
\setlength{\intextsep}{10pt} % Space between text and floats


\numberwithin{equation}{enumi}
\numberwithin{figure}{enumi}
\renewcommand{\thetable}{\theenumi}
\begin{enumerate} [start=16]
    \item Let $\alpha$ \textgreater $0$, $\beta$\textgreater$0$ be such that $\alpha^3+\beta^3=4$.If the maximum value of the term independent of $x$ in the binomial expansion of \brak{\alpha x^\frac{1}{9}+\beta x^\frac{-1}{6}} is $10k$,then $k$ equals to:\\
     \hfill(2020-4Marks)
    \begin{enumerate}
        \item $176$
        \item $336$
        \item $352$
        \item $84$
    \end{enumerate}
    \item Let $S$ be the set of all $\lambda$ $\in$ $R$ for which the system of linear equations \\
    $2x-y+2z$=$2$\\
    $x-2y+\lambda z$=$-4$\\
    $x+\lambda y+z$=$4$ has no solution. Then the set $S$\\
    \hfill(2020-4Marks)
    \begin{enumerate}
        \item is an empty set
        \item is a singleton
        \item contains more than two elements.
        \item contains exactly two elements.
    \end{enumerate}
    \item Let $X$=$\cbrak{x\in N:1\leq x\leq 17}$ and $Y$=$\cbrak{ax+b:x \in X and a,b \in R,a \textgreater0}$.If mean and variance of elements of Y are 17 and 216 respectively then a+b is equal to:\\
    \hfill(2020-4Marks)
    \begin{enumerate}
        \item $27$
        \item $7$
        \item $-7$
        \item $9$
    \end{enumerate}
    \item Let $y$=$y\brak{x}$ be the solution of the differential equation, $\frac{2+\sin{x}}{\brak{y+1}\brak{\frac{dy}{dx}}}$=$-\cos{x}$, $y$\textgreater $0$,$y\brak{0}$=$1$.If $y\brak{\pi}$=$a$,and \brak{\frac{dy}{dx}}at $x$ = $\pi$ is $b$, then the ordered pair \brak{a,b} is equal to:\\
    \hfill(2020-4 Marks)
    \begin{enumerate}
        \item $\brak{2,\frac{2}{3}}$
        \item $\brak{1,1}$
        \item $\brak{2,1}$
        \item $\brak{1,-1}$
    \end{enumerate}
    \item The plane passing through the points $\brak{1,2,1}$, $\brak{2,1,2}$ and parallel to the line, $2x = 3y$, $z = 1$ also passes through the point:\\
    \hfill(2020-4 Marks)
    \begin{enumerate}
        \item $\brak{0,-6,2}$
        \item $\brak{0,6,-2}$
        \item $\brak{-2,0,1}$
        \item $\brak{2,0,-1}$
    \end{enumerate}
    \item  The number of integral values of $k$ for which the line, $3x+4y$ = $k$ intersects the circle, $x^2+y^2-2x-4y+4$=$0$ at two distinct points is\\
    \hfill(2020- Marks)
    \item Let $\Vec{a},\Vec{b}$ and $\Vec{c}$ be three unit vectors such that $\abs{a-b}^2+\abs{a-c}^2=8$.Then $\abs{a+2b}^2+\abs{a+2c}^2$ is equal to:\\
    \hfill(2020-4Marks)
    \item If the letters of the word MOTHER be permuted and all the words so formed  be listed as in a dictionary, then the position of the word MOTHER is \dots \\
    \hfill(2020-4 Marks)
    \item  If $\lim_{x \to 1}$ $\frac{x+x^2+x^3+\dots x^n-n}{x-1}$=$820$ $n \in N$, then the value of $n$ is equal to: \\
    \hfill(2020-4 Marks)
    \item The integral $\int_{0}^{2} \abs{\abs{x-1}-x}dx$ is equal to:\\
    \hfill(2020-4Marks)
    
\end{enumerate}


\end{document}4
